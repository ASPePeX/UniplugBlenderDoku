\mysection{}{Fragestellung} \label{sec:Frage}

Diese Arbeit basiert auf einem vorangehenden Forschungsprojekt, in dem die Programmierschnittstelle der 3D-Grafiksoftware Cinema 4D mit Hilfe des Programmierwerkzeugs \nameref{sec:SWIG} von \CC nach \CS übersetzt und durch Anbindung von FUSEE\footnote{An der Hochschule Furtwangen entwickelte 3D-Echtzeit-Engine (siehe http://fusee3d.org).} ein Export-Plugin realisiert wurde.
In dieser Arbeit soll nun gezeigt werden, wie eine ähnliche Übersetzung der Programmierschnittstelle der 3D-Grafiksoftware Blender möglich ist, mit der gleichzeitigen Betrachtung der zukünftigen Anbindung von weiteren 3D-Softwareprodukten. Die so geschaffene allgemeine Programmierschnittstelle in \CS wird im Weiteren Uniplug genannt.


%Große Softwarefülle, jede Software eigene API. Pluginentwickler beschränkt auf die jeweilige Software.
%Ansatz alles nach \CS dann Logische übersetzung und mithilfe Fusees 3D-Funktionen kram machen.