\mysubsection{Linda Schey, Sarah Häfele}{Swig}\label{sec:SWIG}

Swig ist ein Programmierwerkzeug, das es ermöglicht in C/\CC~geschriebenen Programmcode durch höhere Programmiersprachen anzusprechen. Neben einer Vielzahl von Programmier- und Skriptsprachen unterstützt Swig auch \CS~und eignet sich somit um in \CC~geschrieben Programmcode mit Fusee zu verbinden. Swig wurde gewählt, da es zum einen bereits in dem Cinemea4D-Uniplug von Fusee verwendet wurde und zum anderem einem die Arbeit abnimmt einen kompletten Wrapper von Hand schreiben zu müssen, so dass man sich mittels Swig einen Wrapper generieren lassen kann. Zudem erleichtert es die Wartung von Projekten, da Updates im Code nicht von Hand und mühseliger Suche eingetragen werden müssen, sondern lediglich für die neue Generierung des Wrappers die aktuelle Headerfiles eingebunden werden müssen. Um besondere Datentypen zu übersetzen, die beispielsweise in den Standard APIs von \CC und \CS nicht vorhanden sind oder in einen anderen Datentyp, müssen für diese in einem Swig-Interface File extra Vorschriften definiert werden.
Um den Wrapper zu generieren, wurde zusätzlich die Python Library eingebunden, da das Headerfile teilweise Python Klassen verwendet. Außerdem wurde die Fusee.Math Library referenziert, da einige Datentypen auf Fusee-Datentypen gemapt werden sollen.

\subsubsection{Überblick}
Swig ist nicht intuitiv verwendbar. Bei der Nutzung ist einiges zu beachten, was durch die Swig-Dokumentation nicht gleich ersichtlich ist und erst durch Trial-and-Error langsam erarbeitet werden kann (hier spielen zudem noch das Zusammentreffen von Blender und Fusee mit all ihrer Eigenheiten eine wichtige Rolle, wie später noch besser ersichtlich wird). Deshalb zunächst eine Liste mit Dingen, auf die beim Entwickeln mit Swig zu achten ist und auf deren Inhalt zum Teil später noch näher eingegangen wird.

\begin{itemize}
\item Den Build der Solution sowie deren Unterprojekte auf \textbf{Release} stellen: Da Blender Python \textcolor{red}{als Kommunikationssprache verwendet und Python ...} muss das Projekt als Release gebaut werden, da sonst \textcolor{red}{?????????} Näheres zu den Buildeinstellungen in \textcolor{red}{...}
\item Die richtige \textbf{Build-Reihenfolge} ist unbedingt zu beachten. Zudem sollten zuvor \emph{geswiggte} Klassen aus dem Ordner gelöscht werden, wenn an ihnen Änderungen vorgenommen wurden, um Fehler durch alte Dateien zu vermeiden. Auf die Reihenfolge soll später näher eingegangen werden (siehe \textcolor{red}{???}).
\item \textcolor{red}{BigObj: wieso, weshalb, warum}
\item \textbf{Größe des Headerfiles:} Aus Sichtweise der Swig-Entwickler sind einzelne Headerfiles pro Klassen besser zu swiggen als ein großes mit allen gesammelten Klassen. Das Swiggen dauert mit einem großen File dementsprechend lang zum Bauen und tritt ein Fehler auf, muss der ganze Prozess wiederholt werden. Fehler beim Swiggen werden so erst am Ende erkannt. Zudem vermuten das Swig-Team dieses Projektes, dass weitere Komplikationen durch das zu großes File aufgetreten sind. 
\item Die \textbf{Unterschiede zwischen \CC und \CS} müssen verstanden und beachtet werden, da SWIG nicht alle Klassen, Methoden, etc. einfach in die andere Sprache übersetzen kann. Hierzu zählen sicherlich 
	\begin{itemize}
	\item \textbf{Speicher Allokation}
	\item \textbf{Garbage Collector:} Durh den Managed Code in \CS können eventuell Probleme auftreten. Im unmanaged \CC -Code sind keine Vorbereitungen für einen eventuellen Garbage Collector getroffen. Wird dieser Code mit SWIG nach \CS übersetzt, kann es Probleme bei Dependencies geben, dann nämlich, wenn der Garbage Collector ein Objekt entfernt, auf das ein anderes zugreifen möchte. Hier kann es zu Programmabstürzen kommen. \textcolor{red}{? Lösungsvorschläge weiter unten}
	\item \textbf{Default Werte:} \CC und \CS gehen unterschiedlich mit Default-Werten um, wenn eine Variable nicht initialisiert wurde. \CC vergibt einen völlig beliebigen Wert, während \CS meist feste Default-Werte hat. Dies ist wiederum bei der Übersetzung mit Swig zu beachten. 
	\item \textbf{Pointer:} Da die Anwendung \CC - in \CS -Code umwandeln soll, treten hier verhäuft Fehler und Komplikationen auf. In Unterkapitel \ref{subsubsec:Datentypen} soll dies eingehend behandelt werden. 
	\end{itemize}
\item Die Spezialfälle werden in der SWIG-Doku nicht ausreichend behandelt, weswegen die Entwickler durch so genannte \textbf{Mailing Lists} kontaktiert wurden. Der Dialog mit den Spezialisten bringt neue Erkenntnisse, kann aber nicht alle Spezialfälle klären. Hierzu später ebenfalls mehr.
\item \textbf{Syntax:} Syntaxunterschiede können ebenfalls Komplikationen hervorrufen. Da Swig hauptsächlich Strings ersetzt, muss die Struktur der jeweiligen Syntax immer beachtet werden. So muss zum Beispiel in \CC der Zeigerstern (*) direkt vor dem Variablennamen stehen, wobei in \CS dieser zur Typendeklaration gehört. \textcolor{red}{???}
\item \textbf{Debuggen mit SWIG:} Da es kein Syntax-Highlighting für SWIG in Visual Studio gibt, ist das Finden von Fehlern im Swig-Code nicht leicht. Hierzu hat sich das Anhängen von aussagekräftigen Inline-Kommentaren direkt hinter der zu ersetzenden Swig-Variablen als nützlich erwiesen. Der Kommentar wird so in den erzeugten Code mit eingebaut und es kann explizit danach gesucht werden, um Fehler nachzuverfolgen.
\end{itemize}



	
	
		


	Ignore Genereated\\


\subsubsection{Aufbau des Projektes in VS}\label{subsubsec:Aufbau}

Um das von Fabi erzeugte \CC~Headerfile nach \CS~zu wrappen wurde eine Solution aufgesetzt die aus fünf Projekten Besteht und den \CC~Code nach \CS~wrappt. Im folgenden werden diese Projekte in der Reihenfolge wie sieh für eine erfolgreiche Generierung des \CS~Codes gebaut werden müssen erläutert.\\


\begin{description}
\item[CppApi]\hfill \\
Enthält das \CC~Headerfile mit allen Klassen und Funktionen. Generiert \emph{CppApi.dll}
\item[SWIG]\hfill \\
Enthält das Swig-Interface \emph{CppApi.i} nach dessen Vorschriften besondere Datentypen geswiggt werden (mehr dazu in \ref{subsubsec:swiginterface}). Generiert anhand des Headerfiles entsprechenden Wrapper in \CS. Diese werden dem \emph{CsWrapper} Projekt hinzugefügt.
\item[CppWrapper]\hfill \\
Enthält das \CC~ File CppApiWrapper.cpp, das ebenfalls durch Swig Generiert wird. Es stellt die Schnittsichtelle zwischen dem \CC~ Code und dem \CS~ Code dar unter Berücksichtigung der durch Swig gemarschallten Typen und deren Umwandlungen. Des weiteren wird die \emph{CppWrapper.dll} generiert. 
\item[CsWrapper]\hfill \\
Generiert CsWrapper.dll die die aus dem SWIG Projekt generierten \CS~Klassen enthält. 
\item[CsClient]\hfill \\
Ist lediglich ein Test Projekt mit dem der geswiggte Code in \CS~getestet wird. Im Hinblick auf das Uniplug entspräche dies dem Plugin? -> Ist jetzt Plugin. Die Init-Funktion wird von \CS aus aufgerufen.
\item[ManagedBridge]\hfill \\ Brücke zwischen dem \CS-Plugin und dem \CC-Plugin für Blender. Es handelt sich dabei um ein Managed-\CC-Projekt, das das \CS{}-Plugin referenziert, dessen Funktionen aufruft und die entsprechenden Funktionsaufrufe exportiert, sodass diese von \CC aus aufgerufen werden können.
\item[BlenderPlug]\hfill \\ Das Projekt BlenderPlug ist das eigentliche \CC-Modul, das in Blender als Python-Erweiterungs-Modul geladen werden kann und eine Init-Funktion bereitstellt, um über die ManagedBridge hinweg auf die \CS{}-Funktionalität zuzugreifen. Der relative kurze \CC-Quelltext dieses Programms richtet sich nach den Vorgaben von Python\footnote{\url{https://docs.python.org/3/extending/extending.html}} und muss mittels Python-Compiler kompiliert werden. Ein entsprechendes Script und eine entsprechende Konfigurationsdatei (\emph{setup.py}) werden im Post-Build-Prozess aufgerufen. Die daraus entstehende pyd-Datei (letztlich eine \CC-DLL) wird dann zusammen mit allen anderen Projektdateien in den Unterordner \emph{Fusee} im Addon-Ordner im Blender-Hauptverzeichnis und in das Blender-Hauptverzeichnis selbst kopiert. In Blender kann dann in der Python-Konsole über \emph{from fusee import uniplug} und \emph{uniplug.init()} die Init-Funktion ausgeführt werden. Siehe hierzu auch das angehängte Tutorial zur Erstellung eines eigenen Plugins.
\end{description}

  

\subsubsection{Build-Einstellungen in VS}\label{subsubsec:Build}
	Bigobject (C++->Command Line "`/bigobj"' )\\
	Release evtl.


\subsection{Das Swig-Interface File}\label{subsubsec:swiginterface}
Da in diesem Projekt mit Blender bzw. Fusee Spezifischen Datentypen gearbeitet wird, die weder in der Standard \CC noch in der\CS API gibt mussten diese im Swig-Interface behandelt werden. 

\subparagraph{Arrays}
Swig mappt Arrays von \CC nach \CS standardmäßig als Pointer, so das auf der \CS Seite dann nicht einfach mit einem klassischen \CS-Array gearbeitet werden kann sondern mit einem \emph{SWIGTYPE\_p\_int} oder \emph{SWIGTYPE\_p\_double}. Mit dem vordefinierten Interface \emph{arrays\_csharp.i} gibt es nur die Möglichkeit Arrays namentlich mit dem Befehl \emph{\%apply} in ein \CS-Array zu mappen. Das ist für ein Projekt dieser Größe aber nicht passend, das es extrem aufwändig ist die Namen aller Array herauszusuchen und in das Interface einzutragen und lässt sich nicht automatisieren. Deshalb wurde für Integer, Float und Bolean Arrays Angaben zum Typemapping gemacht, sodass, egal welche Größe ein Array hat und unabhängig vom Namen zu einem \CS Array umgewandelt wird und umgekehrt (siehe Listing\ref{lst:Arrays}). 

\begin{code}[caption={Beispiel Arraymapping},label={lst:Arrays}, escapechar=|]
%typemap(cstype, out="$csclassname") int[ANY] "int[]"
%typemap(csin) int[ANY] " $csinput"
%typemap(imtype) int[ANY] "int[]"
\end{code}

\subparagraph{std::map}

Im Header File treten viele Key-Value Paare Typ \emph{std::map<string, int>}, \emph{std::map<int, string>} oder \emph{std::map<string, DATA\_TYPE\footnote{Beliebiger Datentyp}>} auf. Um dieses Konstrukt in ein Äquivalents in \CS zu konvertieren, wird das \emph{std\_map.i} Interface in das Interface eingebunden werden. Das ermöglicht es diese Key-Value Paare in einen Datentyp zu übersetzen, der die selben Eingenschaften wie ein \CS~ Dictionary. Der Name des Datetypen wir im im Interface definiert und ist dann in \CS unter diesem Damen zu finden. Der so definierte Datentyp funktioniert im Großen und Ganzen wie ein Dictionary (siehe Listing\ref{lst:stdmap} Zeile\ref{line:stringintmap}). Um ein Key-Value Paar, das einen beliebigen Datentyp enthält nach \CS mappen zu können, muss für jeden Datentyp ein extra Template erstellt werden. Um hier zumindest ein wenig Copy-Paste-Arbeit zu sparen wurde per \emph{\%define} ein Template defniert (siehe Listing\ref{lst:stdmap} Zeile\ref{line:stdmapdefine}), das einen Datentyp entgegennimmt, der dann mit einem String zu einem Key-Value paar zusammen gefügt wird. Dieses Template muss dann für jeden Datentyp einmal aufgerufen werden (siehe Listing\ref{lst:stdmap} Zeile\ref{line:stdtemplate}). 

\begin{code}[caption={std::map},label={lst:stdmap}, escapechar=|]
%include "std_map.i"

%template(String_Int_Map)  std::map<std::string, int>;|\label{line:stringintmap}|
	
%define %std_templates(DATA_TYPE)|\label{line:stdmapdefine}|
%template(String_ ## DATA_TYPE ## _Map) std::map<std::string, UniplugBL::DATA_TYPE>;
%enddef

%std_templates(Property);|\label{line:stdtemplate}|
%std_templates(Function);
%std_templates(EnumPropertyItem);
\end{code}

\subparagraph{std::vector}

Im Header-File treten drei Fälle des \emph{std::vector<T>} auf. Diese können mit wenig Aufwand unter Verwendung des \emph{std\_vector.i} Interfaces einfach gemappt werden (siehe Listing \ref{lst:stdvector})

\begin{code}[caption={std::map},label={lst:stdvector}, escapechar=|]
%include "std_vector.i"

%template(IntVector) std::vector<int>;
%template(FloatVector) std::vector<float>;
%template(DoubleVector) std::vector<double>;
\end{code}

\subparagraph{ignore-Regex}

\subparagraph{POD Datentypen}


%\subsubsection{Typemaps}\label{subsubsec:Typemaps}

%Was sind Typemaps und wie funktionieren sie?\\

%\subsubsection{Besondere Datentypen}\label{subsubsec:Datentypen}
%%das hier gehört dann zu POD Datentypen
%Besondere Datentypen sind jene, die nicht zu den Standard Datentypen in \CS~gewrappt werden, oder gar keine Standarddatentypen sind. Hier sind das spezielle Typen zu Darstellung von Vektoren oder Matrizen im 3-dimensionalen Raum. Dafür wurden verschiedene Datentypen ausprobiert. Zu Beginn wurde begonnen einen 3-dimensionalen Vektor als Feld mit drei Elementen darzustellen. Dafür wurde im Swig-Interface CppApi.i ein extra Typemap angelegt, das diese dreielementige Feld in den Fusse Datyntyp float3 mapt. Da hier aber Probleme auftraten, da Felder in \CC~nicht als Rückgabe wert einer Funktion zulässig sind, sondern nur als Pointer, \CS~aber keine Pointer unterstützt, wurde die der Datentyp bei der Erstellung des Headerfiles in den Datentyp std::array<float, 3> geändert. Auch für diesn Datentyp wurde zunächst ein entsprechendes Typemap erstellt, das teilweise funktionierte. Jedoch wurde bei der Verwendung eines std::array<> als Parameter einer Funktion übergeben dann ein Ecxeption geworfen, da für std::array  NULL bzw. 0 zurückgegeben werden kann, was bei einer Standard-Exceptio in Swig der Fall ist. So wurde wieder der Datentyp geändert, diesmal wurde ein extra Struct definiert, das die Eigenschaften eines \enquote{Plain old Datatype} aufweist, so dass anstatt eines Objektes auch NULL zurück gegeben werden kann. Über die Swig-User-Mailingliste kam letztendlich der Hinwei,d as im Typemap festzulegen ist, das für der Parameter \enquote{out} außerdem festlegen kann, was als \enquote{null} zurückgeben wird, hier ein leeres Objekt des Strucs. Dies Lösung hätte auch für das std::array funktioniert, so das nicht erst ein extra neuer Datentyp eingeführte werden müsste. Es wurde aber beim Struct geblieben, da im Hinblick auf das Ziel ein Uniplug zu erstellen, man so bei der Erstellung des Headefiles, von den unterschiedlichen Modellierungsprogrammen erst die verschiedenen Datentypen die Vektoren oder Matrizen darstellen in einen einheitlich \CC~Typ umwandeln kann, so das hier das Typemap für jedes Modellierungsprogramm das selbe wäre. 
%Datentypen die im aktuellen \emph{CppApi.i} Interface mit Typemaps nach \CS~und dem entsprechenden Typ in Fusee gemapt werden sind:\\
%\begin{itemize}
	%\item FVector2 nach Fusee.Math.float2
	%\item FVector3 nach Fusee.Math.float3
	%\item FVector4 nach Fusee.Math.float4
	%\item VFLOAT16 nach Fusee.Math.float4x4
%\end{itemize}
%
		%- std::vector, std::map\\
	

Zeigertypen können in C\# nur im unsafe-Kontext verwendet werden. Zudem unterliegen sie weiteren Einschränkungen (Boxing und Unboxing wird nicht unterstützt und sie erben nicht von object). Diese und weitere Gründe bewegen dazu, mit SWIG aus den c++-Pointern andere, entsprechende Typen in C\# zu generieren. Dazu müssen die Objekte, die hinter den betreffenden Klassen stehen, analysiert werden. Was repräsentiert der Objekttyp in Blender? Wie ist dieses Konzept in Fusee umgesetzt? \textcolor{red}{Evtl hier noch ein Beispiel mit Fusee Float3 oder so. Danach typemaps dazu}

--> nicht nullbar, wird aber wie Pointer behandelt. Exception-Fehler usw.




\subsubsection{GC}\label{subsubsec:GC}
