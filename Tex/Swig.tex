\mysubsection{Linda Schey, Sarah Häfele}{Swig}\label{sec:SWIG}

Swig ist ein Programmierwerkzeug, das es ermöglicht in C/\CC~geschriebenen Programmcode durch höhere Programmiersprachen anzusprechen. Neben einer Vielzahl von Programmier- und Skriptsprachen unterstützt Swig auch \CS~und eignet sich somit um in \CC~geschrieben Programmcode mit Fusee zu verbinden. Swig wurde gewählt, da es zum Einen bereits in dem Cinemea4D-Uniplug von Fusee verwendet wurde und zum Anderen einem die Arbeit abnimmt einen kompletten Wrapper von Hand schreiben zu müssen, sodass man sich mittels Swig einen Wrapper generieren lassen kann. Zudem erleichtert es die Wartung von Projekten, da Updates im Code nicht von Hand und mühseliger Suche eingetragen werden müssen, sondern lediglich für die neue Generierung des Wrappers die aktuellen Header-Dateis eingebunden werden müssen. Um besondere Datentypen zu übersetzen, die beispielsweise in den Standard APIs von \CC und \CS nicht vorhanden sind oder in einen anderen Datentyp, müssen für diese in einem Swig-Interface File extra Vorschriften definiert werden.
Um den Wrapper zu generieren, muss zusätzlich die Python Library eingebunden werden, da das Header-Datei teilweise Python Klassen verwendet. Außerdem wird die Fusee.Math Library referenziert, da einige Datentypen auf Fusee-Datentypen gemappt werden.

\subsubsection{Überblick}
Swig ist nicht intuitiv verwendbar. Bei der Nutzung ist einiges zu beachten, was durch die Swig-Dokumentation nicht gleich ersichtlich ist und erst durch Trial-and-Error langsam erarbeitet werden kann (hier spielt zudem noch das Zusammentreffen von Blender und Fusee mit all ihren Eigenheiten eine wichtige Rolle, wie später noch besser ersichtlich wird). Deshalb zunächst eine Liste mit Dingen, auf die beim Entwickeln mit Swig zu achten ist und auf deren Inhalt zum Teil später noch näher eingegangen wird.

\begin{itemize}
%\item Den Build der Solution sowie deren Unterprojekte auf \textbf{Release} stellen: Da Blender Python \textcolor{red}{als Kommunikationssprache verwendet und Python ...} muss das Projekt als Release gebaut werden, da sonst \textcolor{red}{?????????} Näheres zu den Buildeinstellungen in \textcolor{red}{...}-> kann glaub weg
\item Die richtige \textbf{Build-Reihenfolge} ist unbedingt zu beachten. Zudem sollten zuvor \emph{geswiggte} Klassen aus dem Ordner gelöscht werden, wenn an ihnen Änderungen vorgenommen wurden, um Fehler durch alte Dateien zu vermeiden. Auf die Reihenfolge soll später näher eingegangen werden (siehe Abschnitt \ref{subsubsec:Aufbau}).
%\item \textcolor{red}{BigObj: wieso, weshalb, warum} -> brauchne wir nicht mehr
\item \textbf{Größe des Header-Dateis:} Aus Sichtweise der Swig-Entwickler sind einzelne Header-Dateis pro Klassen besser zu swiggen als ein Großes mit allen gesammelten Klassen. Das Swiggen dauert mit einem großen File dementsprechend lang um zu bauen und tritt ein Fehler auf, muss der ganze Prozess wiederholt werden. Fehler beim Swiggen werden so erst am Ende erkannt. %Zudem vermuten das Swig-Team dieses Projektes, dass weitere Komplikationen durch das zu großes File aufgetreten sind. 
\item Die \textbf{Unterschiede zwischen \CC und \CS} müssen verstanden und beachtet werden, da Swig nicht alle Klassen, Methoden, etc. einfach in die andere Sprache übersetzen kann. Hierzu zählen sicherlich 
	\begin{itemize}
	\item \textbf{Speicher Allokation und Pointer:}
	Da die Anwendung \CC - in \CS -Code umwandeln soll, können so Fehler bei der Speicher Belegung auftreten. Um dies so weit als möglich von vorneherein auszuschließen, wurde auf die Verwendung von Pointern verzichtet.
	\item \textbf{Garbage Collector:} Durch den Managed Code in \CS können eventuell Probleme auftreten. Im unmanaged \CC -Code sind keine Vorbereitungen für einen eventuellen Garbage Collector getroffen. Wird dieser Code mit SWIG nach \CS übersetzt, kann es Probleme bei Dependencies geben, dann nämlich, wenn der Garbage Collector ein Objekt entfernt, auf das ein anderes zugreifen möchte. Hier kann es zu Programmabstürzen kommen. \textcolor{red}{? Lösungsvorschläge weiter unten}
	\item \textbf{Default Werte:} \CC und \CS gehen unterschiedlich mit Default-Werten um, wenn eine Variable nicht initialisiert wurde. \CC vergibt einen völlig beliebigen Wert, während \CS meist feste Default-Werte hat. Dies ist wiederum bei der Übersetzung mit Swig zu beachten. Dieses Problem entstand beim Übersetzten von Klassen, die nicht dem Schema eines PODs entsprechen (vgl. Abschnitt \ref{besonderedatentypen}).
	%\item \textbf{Pointer:} Da die Anwendung \CC - in \CS -Code umwandeln soll, treten hier verhäuft Fehler und Komplikationen auf. In Unterkapitel \ref{subsubsec:Datentypen} soll dies eingehend behandelt werden. ich glaub das kann dann weg
	\end{itemize}
\item Spezialfälle werden in der Swig-Doku nicht ausreichend behandelt, weswegen die Entwickler durch sogenannte \textbf{Mailing Lists} kontaktiert wurden. Der Dialog mit den Spezialisten bringt neue Erkenntnisse, kann aber nicht alle Spezialfälle klären. %Hierzu später ebenfalls mehr.
\item \textbf{Syntax:} Syntaxunterschiede können ebenfalls Komplikationen hervorrufen. Da Swig hauptsächlich Strings ersetzt, muss die Struktur der jeweiligen Syntax immer beachtet werden. So muss zum Beispiel in \CC der Zeigerstern (*) direkt vor dem Variablennamen stehen, wobei in \CS dieser zur Typendeklaration gehört. \textcolor{red}{??? Linda: ja da weiß ich auch noch nciht od das wirklich relevant is oder was da dann rein soll}
\item \textbf{Debuggen mit Swig:} Da es kein Syntax-Highlighting für SWIG in Visual Studio gibt, ist das Finden von Fehlern im Swig-Code nicht leicht. Hierzu hat sich das Anhängen von aussagekräftigen Inline-Kommentaren direkt hinter der zu ersetzenden Swig-Variablen als nützlich erwiesen. Der Kommentar wird so in den erzeugten Code mit eingebaut und es kann explizit danach gesucht werden, um Fehler nachzuverfolgen.
\end{itemize}

\subsubsection{Aufbau des Projektes in VS}\label{subsubsec:Aufbau}

Um das \CC~Header-Datei nach \CS~zu wrappen wurde eine Solution aufgesetzt die aus fünf Projekten besteht und den \CC~Code nach \CS~wrappt. Im Folgenden werden diese Projekte, in der Reihenfolge wie sieh für eine erfolgreiche Generierung des \CS~Codes gebaut werden müssen, erläutert.\\

\begin{description}
\item[CppApi]\hfill \\
Enthält das \CC~Header-Datei mit allen Klassen und Funktionen. Generiert \emph{CppApi.dll}
\item[SWIG]\hfill \\
Enthält drei Swig-Interfaces (\emph{CppApi.i, POD\_Mapping.i und \\
uniplug\_blender\_api\_swig.i}) nach dessen Vorschriften besondere Datentypen geswiggt werden (mehr dazu in Abschnitt \ref{subsubsec:swiginterface}). Anhand der Header-Datei werden die entsprechenden Wrapper in \CS generiert. Diese werden dem \emph{CsWrapper} Projekt hinzugefügt.
\item[CppWrapper]\hfill \\
Enthält das \CC~ File CppApiWrapper.cpp, das ebenfalls durch Swig Generiert wird. Es stellt die Schnittsichtelle zwischen dem \CC~ Code und dem \CS~ Code dar unter Berücksichtigung der durch Swig gemarschallten Typen und deren Umwandlungen. Des Weiteren wird die \emph{CppWrapper.dll} generiert.
\item[CsWrapper]\hfill \\
Generiert CsWrapper.dll die die aus dem SWIG Projekt generierten \CS~Klassen enthält. 
\item[CsClient]\hfill \\
Dieses Projekt ist das Plugin, das der Entwickler schreiben kann. Hier wird die Init-Funktion von \CS aus nach Blender aufgerufen.
\item[ManagedBridge]\hfill \\ Brücke zwischen dem \CS-Plugin und dem \CC-Plugin für Blender. Es handelt sich dabei um ein Managed-\CC-Projekt, das das \CS-Plugin referenziert, dessen Funktionen aufruft und die entsprechenden Funktionsaufrufe exportiert, sodass diese von \CC aus aufgerufen werden können.
\item[BlenderPlug]\hfill \\ Das Projekt BlenderPlug ist das eigentliche \CC-Modul, das in Blender als Python-Erweiterungs-Modul geladen werden kann und eine Init-Funktion bereitstellt, um über die ManagedBridge hinweg auf die \CS{}-Funktionalität zuzugreifen. Der relative kurze \CC-Quelltext dieses Programms richtet sich nach den Vorgaben von Python\footnote{\url{https://docs.python.org/3/extending/extending.html}} und muss mittels Python-Compiler kompiliert werden. Ein entsprechendes Script und eine entsprechende Konfigurationsdatei (\emph{setup.py}) werden im Post-Build-Prozess aufgerufen. Die daraus entstehende pyd-Datei (letztlich eine \CC-DLL) wird dann zusammen mit allen anderen Projektdateien in den Unterordner \emph{Fusee} im Addon-Ordner im Blender-Hauptverzeichnis und in das Blender-Hauptverzeichnis selbst kopiert. In Blender kann dann in der Python-Konsole über \emph{from fusee import uniplug} und \emph{uniplug.init()} die Init-Funktion ausgeführt werden. Siehe hierzu auch das angehängte Tutorial zur Erstellung eines eigenen Plugins.
\end{description}

\subsubsection{Swig-Interface Files}\label{subsubsec:swiginterface}
Da in diesem Projekt mit Blender- bzw. Fuseespezifischen Datentypen gearbeitet wird, die weder in der Standard \CC noch in der\CS API gibt, müssen diese im Swig-Interface behandelt werden. Neben diesen besonderen Datentypen müssen auch Andere durch Anweisungen angepasst werden, sodass sie in \CS später verwendbar sind. In drei Interface Files sind mehrere Typemaps und Mappingvorschriften definiert. \emph{POD\_Mapping.i} enthält ein Typemap für die Übersetzung von besonderen Datentypen (siehe Abschnitt \ref{besonderedatentypen}). Das \emph{CppAPi.i} ist gemischt, hier werden kleine Typemaps definiert, sowie verschidene Defines die nur über wenige Zeilen gehen und kein einzelnes Interface File benötigen. Im File \emph{uniplug\_blender\_api\_swig.i} werden alle Klassen der Header-Datei die in einem Typemap namentlich aufgerufen werden müssen auf diese Klassen ausgeführt.

\subparagraph{Arrays}
Swig mappt Arrays von \CC nach \CS standardmäßig als Pointer, so das auf der \CS Seite dann nicht einfach mit einem klassischen \CS-Array gearbeitet werden kann sondern mit einem \emph{SWIGTYPE\_p\_int} oder \emph{SWIGTYPE\_p\_double}. Mit dem vordefinierten Interface \emph{arrays\_csharp.i} gibt es nur die Möglichkeit Arrays namentlich mit dem Befehl \emph{\%apply} in ein \CS-Array zu mappen. Das ist für ein Projekt dieser Größe aber nicht passend, das es extrem aufwändig ist die Namen aller Array herauszusuchen und in das Interface einzutragen und lässt sich nicht automatisieren. Deshalb wird für Integer, Float und Bolean Arrays Angaben zum Typemapping gemacht, sodass, egal welche Größe ein Array hat und unabhängig vom Namen zu einem \CS Array umgewandelt wird und umgekehrt (siehe Listing \ref{lst:Arrays}). Aktuell sind jedoch keine Arrays mehr in der Header-Datei enthalten, da sie alle durch den entsprechende Datentyp (VLFOAT3, VLOAT4, VINT3, ...) abgedeckt werden.

\begin{code}[caption={Beispiel Arraymapping},label={lst:Arrays}, escapechar=|]
%typemap(cstype, out="$csclassname") int[ANY] "int[]"
%typemap(csin) int[ANY] " $csinput"
%typemap(imtype) int[ANY] "int[]"
\end{code}

\subparagraph{std::map}

In der Header-Datei treten viele Key-Value Paare vom Typ \emph{std::map<string, int>}, \emph{std::map<int, string>} oder \emph{std::map<string, DATA\_TYPE\footnote{Beliebiger Datentyp}>} auf. Um dieses Konstrukt in ein Äquivalentes in \CS zu konvertieren, wird das \emph{std\_map.i} Interface in das Interface eingebunden werden. Das ermöglicht es diese Key-Value Paare in einen Datentyp zu übersetzen, der die selben Eigenschaften  wie ein \CS Dictionary besitzt.  Der Name des Datentypen wir im Interface definiert und ist dann in \CS unter diesem Damen zu finden. Der so definierte Datentyp funktioniert im Großen und Ganzen wie ein Dictionary (siehe Listing\ref{lst:stdmap} Zeile\ref{line:stringintmap}). Um ein Key-Value Paar, das einen beliebigen Datentyp enthält nach \CS mappen zu können, muss für jeden Datentyp ein extra Template erstellt werden. Um hier zumindest ein wenig Copy-Paste-Arbeit zu sparen, wurde per \emph{\%define} ein Template definiert (siehe Listing\ref{lst:stdmap} Zeile\ref{line:stdmapdefine}), das einen Datentyp entgegennimmt, der dann mit einem String zu einem Key-Value paar zusammengefügt wird. Dieses Template muss dann für jeden Datentyp einmal aufgerufen werden (siehe Listing \ref{lst:stdmap} Zeile \ref{line:stdtemplate}). 

\begin{code}[caption={std::map},label={lst:stdmap}, escapechar=|]
%include "std_map.i"

%template(String_Int_Map)  std::map<std::string, int>;|\label{line:stringintmap}|
	
%define %std_templates(DATA_TYPE)|\label{line:stdmapdefine}|
%template(String_ ## DATA_TYPE ## _Map) std::map<std::string, UniplugBL::DATA_TYPE>;
%enddef

%std_templates(Property);|\label{line:stdtemplate}|
%std_templates(Function);
%std_templates(EnumPropertyItem);
\end{code}

\subparagraph{std::vector}

In der Header-Datei treten drei Fälle des \emph{std::vector<T>} auf. Diese können mit wenig Aufwand unter Verwendung des \emph{std\_vector.i} Interfaces einfach gemappt werden (siehe Listing \ref{lst:stdvector})

\begin{code}[caption={std::vector},label={lst:stdvector}, escapechar=|]
%include "std_vector.i"

%template(IntVector) std::vector<int>;
%template(FloatVector) std::vector<float>;
%template(DoubleVector) std::vector<double>;
\end{code}

\subparagraph{ignore-Regex}
Einige Funktionen der Header-Datei müssen ignoriert werden. Dafür stellt Swig die sog. \emph{\%ignore} Anweisung zu zur Verfügung, mit der man jede beliebige Funktion oder ganze Klassen ignorieren, kann, das heißt, sie werden nicht in den Wrapper übernommen. Der Syntax lautet: Zuerst die Ignore-Anweisung und dann der Name der zu ignorierenden Funktion. Beispielsweise \emph{\%ignore examplefunction;}. Die Funktionen \emph{examplefunction} würde im Wrapper dann nicht existieren. Hier ist der Fall jedoch etwas speziell, da nicht eine bestimmte Funktion ignoriert werden soll, sonder alle Funktionen, die ein bestimmtes Namensschema aufweisen. Um das zu erzielen wird Patternmaching mit Regulären Ausdrücken angewandt. Für die Regulären Ausdrücke wird das Perl Schema (PCRE) verwendet. Um ein Pattern zu erkennen und zu ignorieren weicht der Syntax aber von dem Obengenannten ab. Hier muss mit der \emph{\%rename}-Anweisung, einem Regulären Ausdruck und dem Zusatz eines \emph{\$ignore}-Value gearbeitet werden, da \emph{ignore} eigentlich nur ein Sonderfall der \emph{\%rename} Anweisung ist.
Listing \ref{lst:ignore} zeit wie alle Funktionen die entweder \emph{sting\_to} oder \emph{to\_string} enthalten ignoriert werden.

\begin{code}[caption={inoreregex},label={lst:ignore}, escapechar=|]
%rename("%(regex:/(\w*)string_to(\w*)/$ignore/)s") "";
%rename("%(regex:/(\w*)to_string(\w*)/$ignore/)s") "";
\end{code}

\subparagraph{Besondere Datentypen} \label{besonderedatentypen}
Die Header-Datei enthält Datentypen die Arrays oder Vektoren oder Matrizen im 3-dimensionalen Raum repräsentieren, die auf die entsprechenden Datentypen in Fusee gemappt werden sollen. Dafür wurden verschiedene Datentypen ausprobiert. Zu Beginn wurde begonnen einen 3-dimensionalen Vektor als Feld mit drei Elementen getestet. 
Der erste Versuch einfache Arrays zu verwenden wurde wieder verworfen, da diese nur als Pointer zurückgegeben werden können und dies wie in Kapitel \ref{sec:header} vermieden werden sollte. Darauf hin wurden diese Datentypen in der Header-Datei auf den Datentyp \emph{std::array<>} geändert. Leider gab es auch in diesem Fall Probleme den Datentyp zu mappen, da seine Eigenschaften nicht denen eines POD entspricht und es so Probleme mit Übergabeparametern diese Typs gibt, da bei Exeptions dieser Typ nicht auf NULL bzw. 0 gecastet werden kann.
Deshalb wurden in der Header-Datei Datentypen definiert, die für sämtliche int-, bool- und float-Arrays verwendet werden. Für diese Art von Datentypen wurden Typemaps geschrieben und auf  das entsprechende Äquivalent in Fusee gemappt. Die Typemaps wurden mit der \emph{\#define}-Anweisung definiert, sodass nicht jeder Datentyp  extra Typemaps benötigt, sonder diese ähnlich wie bei den \emph{std::map<>} teilweise automatisch übersetzt werden und ein Teil der Tipparbeit und Zeilen Code gespart werden kann.
Das \emph{out}-Typemap (siehe Listing \ref{lst:pods}) weißt eine Besonderheit auf, die es theoretisch ermöglicht würde auch nicht PODs zu übersetzen, da im Falle einer Exception nicht NULL bzw. 0 zurückgegeben wird, sondern ein leeres Objekt des Datentyps \enquote{\emph{null="TYPE\_NAME()"}} (vgl. \ref{lst:pods}, Zeile 1). Diese Lösung wurde jedoch erst sehr spät gefunden, sodass letztendlich bei den neu definierten Datentypen nach dem VFLAOT Prinzip behalten wurden, da so auch alle auftretenden Arrays abgedeckt werden und auf das oben beschrieben Arraymapping verzichtet werden kann. 
\begin{code}[caption={Stark vereinfachtes Beispiel eines Typemaps für VFOAT},label={lst:pods}, escapechar=|]
%define %fusee_pod_typemaps(DATA_TYPE, TYPE_NAME, RESULT)
	%ignore "operator []";
	%ignore TYPE_NAME;
	%typemap(cstype, out="RESULT") TYPE_NAME, TYPE_NAME *
		"RESULT"
		%typemap(imtype, out="RESULT") TYPE_NAME, TYPE_NAME *
		"RESULT /* imtype */"
	%typemap(ctype, out="TYPE_NAME") TYPE_NAME, TYPE_NAME *
		"TYPE_NAME /* ctype */"		
		...
	%typemap(out, null="TYPE_NAME() /* out (null) */") TYPE_NAME
	%{ 
		$result = $1;
	%}
	%typemap(in) TYPE_NAME
	%{
		$1 = *((TYPE_NAME *)&($input));
	%}
	%typemap(csin) TYPE_NAME, TYPE_NAME *
		"$csinput"
	%typemap(csdirectorin, pre="") TYPE_NAME, TYPE_NAME *
		"$iminput"
	%typemap(csdirectorout) TYPE_NAME, TYPE_NAME *
		"$cscall"
	%typemap(csvarin) TYPE_NAME, TYPE_NAME *
	%{
			...
	%}
	%typemap(csvarout) TYPE_NAME, TYPE_NAME *
	%{
			...
	%}
%enddef	

%include "POD_Mapping.i"
%fusee_pod_typemaps(float, UniplugBL::VFLOAT3, Fusee.Math.float3);
\end{code}
